\section{Introducción}\label{Intro}

\setbeamercolor{block title}{bg=blue!50!black,fg=white}
\setbeamercolor{block body}{bg=blue!20,fg=white}

\begin{frame}
    \begin{columns}[t]
        \begin{column}{.5\textwidth}
          \tableofcontents[sections={1-2},currentsection]
        \end{column}
        \begin{column}{.5\textwidth}
          \tableofcontents[sections={3-4},currentsection]
        \end{column}
    \end{columns}
\end{frame}

\subsection{¿Qué es \textbf{Moogle!} ?}


\begin{frame}{¿Qué es \textbf{Moogle!} ?}
\begin{center}
    \textbf{Moogle!} es una aplicación \emph{*totalmente original*} cuyo propósito es buscar inteligentemente
    un texto en un conjunto de documentos.\\
    \pause
    Es una aplicación web, desarrollada con tecnología .NET Core 6.0, específicamente usando 
    Blazor como \emph{*framework*} web para la interfaz gráfica, y en el lenguaje C\#.
\end{center}
\end{frame}

\subsection{Arquitectura del sistema}

\begin{frame}{Arquitectura del sistema}
\begin{center}
    \begin{block}{Clases modificadas:}
            \begin{itemize}
                \item[-]<1->{Moogle.cs}
                \item[-]<2->{Program.cs} 
                \item[-]<3->{Index.razor} 
                \item[-]<4->{Index.razor.css}    
            \end{itemize}
    \end{block}

    \pause

    \begin{block}{Clases implementadas por mí:}
            \begin{itemize}
                \item[-]<5->{Document.cs} 
                \item[-]<6->{Operations.cs} 
                \item[-]<7->{QueryOperations.cs}   
            \end{itemize}
    \end{block}
\end{center}
\end{frame}